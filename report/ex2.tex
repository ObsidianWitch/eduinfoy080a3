\section{Exercise 2: Integration with Legacy Software}

First of all, the json database backend from the previous assignment was
extended in order to be able to search for books with it. The
\hl{RegularJSONDatabase.findRecords()} method was implemented and looks for
book records inside the \hl{web\_portal/DB/JSON/books.json} file.\\

Then, a new database layer was added to handle web services. The
\hl{softarch.portal.db.webservice. WebServiceDatabase} and
\hl{.RegularWebServiceDatabase} classes were created. The code for the
\hl{RegularWebServiceDatabase.findRecords()} method uses classes
automatically generated from the \hl{LibrarySearchArtifacts.wsdl} file.\\

Generated classes were created inside the \emph{librarysearch.soft} package. We
instantiate a \hl{LibrarySearchSOAPBindingStub} with a
\hl{LibrarySearchServiceLocator} and the web service url (stored in the portal
\hl{web\_portal.cfg} configuration file). This stub class can process a request
(\hl{StringHolder}) and returns an array of \hl{librarysearch.soft.Book}. Once
these books are retrieved, they can be transformed to the
\hl{softarch.portal.data.Book} class.\\

The newly created \hl{RegularDatabase} implementation is instantiated in
the \hl{DatabaseFacade} class and used in the \hl{findRecords()} method.
Results from both the local (SQL or JSON) and web service databases are
concatenated and returned.\\

\begin{framewarning}
    The NationalLibrary and SoftLibrary web services only contain dummy data.
    Moreover, their search functions do not search, but returns all the results
    without filtering anything (see
    \hl{SoftLibrarySoapBindingImpl.searchBooks()}
    and \hl{LibraryService.searchForBooks()} methods).
\end{framewarning}

\newpage
